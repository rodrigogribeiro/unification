\subsection{Soundness and Completeness Proof}\label{soundness}

Given an arbitrary list of constraints, the unification algorithm
either fails or returns its most general unifier. We have the
following properties:
\begin{itemize}
    \item Soundness: the substitution produced is a unifier of the constraints.

    \item Completeness: the returned substitution is the least
      unifier, according to the substitution ordering defined in
      Section \ref{substitution}.
\end{itemize}

A substitution $S$ is called a unifier of a list of constraints
$\mathbb{C}$ according to whether $\unifier(\mathbb{C},S)$ is provable,
where $\unifier(\mathbb{C},S)$ is defined by induction on $\mathbb{C}$
as follows:

\[ \begin{array}{lcl}
     \unifier([],S)                             & = & \True \\
     \unifier((\tau\eeq\tau') :: \mathbb{C}',S) & = & S(\tau) = S(\tau') \land \unifier(\mathbb{C}',S)
  \end{array} 
\]

\newcommand{\least}{\text{\it least\/}}

A substitution $S$ is a most general unifier of a list of constraints
$\mathbb{C}$ if, for any other unifier $S'$ of $\mathbb{C}$, there
exists $S_1$ such that $S' = S_1 \circ S$; formally:

\[
  \least(S,\mathbb{C}) = 
    \forall S'.\,\unifier(\mathbb{C},S') \to 
       \exists S_1.\, \forall \alpha.\,(S_1 \circ S)(\alpha) = S'(\alpha)
\]
The type of the unification function is a dependent type that ensures
the following property of the returned substitution $S$:

\[ \bigl(\unifier(\mathbb{C},S) \land 
    \least(S,\mathbb{C})\bigr) \lor \text{UnifyFailure($\mathbb{C}$)}
\]
where UnifyFailure($\mathbb{C}$) is a type that encodes the reason why
unification of $\mathbb{C}$ fails. There are two possible causes of
failure: 1) an occurs check error, 2) an error caused by trying to
unify distinct type constructors.

In the formalization source code, the definition of the $\unify$ function
contains ``holes''\footnote{A
  hole in a function definition is a subterm that is left
  unspecified. In Coq, holes are represented by underscores and such
  unspecified parts of a definition are usually filled by tactic
  generated terms.} to mark positions where proof terms are
expected. Instead of writing such proof terms, we left them
unspecified and use tactics to fill them with appropriate proofs. In
the companion source code, the unification function is full of such
holes and they mark the position of proof obligations for soundness,
completeness and termination for each equation of the definition of
$\unify$.

In order to prove soundness obligations we define several small lemmas
that are direct consequences of the definition of the application of
substitutions, which are omitted for brevity. Other lemmas necessary
to ensure soundness are sketched below. They specify properties of
unification and application of substitutions.

\begin{Lemma}\label{unifyty}
For all type variables $\alpha$, types $\tau,\tau'$ and substitutions
$S$, if $S(\alpha) = S(\tau')$ then $S(\tau) = S([\alpha \mapsto
  \tau']\,\tau)$.
\end{Lemma}
\begin{proof}
Induction over the structure of $\tau$.
\end{proof}

\begin{Lemma}
For all type variables $\alpha$, types $\tau$, variable contexts
$\mathcal{V}$ and constraint sets $\mathbb{C}$, if $S(\alpha) = S(\tau)$
and $\unifier(\mathbb{C},S)$ then
$\unifier([\alpha\mapsto\tau]\,\mathbb{C},S)$.
\end{Lemma}
\begin{proof}
Induction over $\mathbb{C}$ using Lemma \ref{unifyty}.
\end{proof}

Completeness proof obligations are filled by scripted automatic proof tactics
using Lemma \ref{extsubst}.

%\section{Automating Proofs}\label{tactics}

Most parts of most proofs used to prove properties of programming
languages and of algorithms are exercises that consist of a lot of
somewhat tedious steps, with just a few cases representing the core
insights. It is not unusual for mechanized proofs to take significant
amounts of code on uninteresting cases and quite significant effort on
writing that code. In order to deal with this problem in our
development, we use $\mathcal{L}$tac, Coq's domain specific language
for writing custom tactics, and Coq built-in automatic tactic
\texttt{auto}, which implements a Prolog-like resolution proof
construction procedure using hint databases within a depth limit.

The main $\mathcal{L}$tac custom tactic used in our development is a
proof state simplifier that performs several manipulations on the
hypotheses and on the conclusion. It is defined by means of two
tactics, called \texttt{mysimp} and \texttt{s}. Tactic \texttt{mysimp}
tries to reduce the goal and repeatedly applies tactic \texttt{s} to
the proof state until all goals are solved or a failure occurs.

Tactic \texttt{s}, shown in Figure \ref{simptac}, performs pattern
matching on a proof state using $\mathcal{L}$tac \texttt{match goal}
construct. Patterns have the form:
\[ \texttt{[h$_1$ : t$_1$,h$_2$ : t$_2$ ... |- C ] => tac}\] 
where each of \texttt{t}$_i$ and \texttt{C} are expressions, which
represents hypotheses and conclusion, respectively, and \texttt{tac}
is the tactic that is executed when a successful match
occurs. Variables with question marks can occur in $\mathcal{L}$tac
patterns, and can appear in \texttt{tac} without the question
mark. Names \texttt{h}$_i$ are binding occurrences that can be used in
\texttt{tac} to refer to a specific hypothesis. Another aspect
worth mentioning is keyword \texttt{context}. Pattern matching with
\texttt{context[e]} is successful if \texttt{e} occurs as a
subexpression of some hypothesis or in the conclusion. In Figure
\ref{simptac}, we use \texttt{context} to automate case analysis on
equality tests on identifiers and natural numbers, as shown below
\begin{lstlisting}
 [ |- context[eq_id_dec ?a ?b] ] => 
          destruct (eq_id_dec a b) ; subst ; try congruence
\end{lstlisting}
Tactic \texttt{destruct} performs case analysis on a term,
\texttt{subst} searchs the context for a hypothesis of the form
\texttt{x = e} or \texttt{e = x}, where \texttt{x} is a variable and
\texttt{e} is an expression, and replaces all occurrences of
\texttt{x} by \texttt{e}.
% Hm? And if e is a program var (say y) ??
Tactic \texttt{congruence} is a decision
procedure for equalities with uninterpreted functions and data type
constructors~\cite{Bertot04}.
\begin{figure}[H]
\begin{lstlisting}
Ltac s :=
  match goal with
    | [ H : _ /\ _ |- _] => destruct H
    | [ H : _ \/ _ |- _] => destruct H
    | [ |- context[eq_id_dec ?a ?b] ] => 
             destruct (eq_id_dec a b) ; subst ; try congruence
    | [ |- context[eq_nat_dec ?a ?b] ] => 
             destruct (eq_nat_dec a b) ; subst ; try congruence
    | [ x : (id * ty)%type |- _ ] => 
             let t := fresh "t" in destruct x as [x t]
    | [ H : (_,_) = (_,_) |- _] => inverts* H
    | [ H : Some _ = Some _ |- _] => inverts* H
    | [ H : Some _ = None |- _] => congruence
    | [ H : None = Some _ |- _] => congruence
    | [ |- _ /\ _] => split
    | [ H : ex _ |- _] => destruct H
  end.

Ltac mysimp := repeat (simpl; s) ; simpl; auto with arith.
\end{lstlisting}
\caption{Main proof state simplifier tactic.}
\label{simptac}
\end{figure}

Tactic \texttt{inverts* H} generates necessary conditions used to
prove \texttt{H} and afterwards executes tactic
\texttt{auto}.\footnote{This tactic is defined on a tactic library
  developed by Arthur Charguéraud~\cite{Pierce:SF}.} Tactic
\texttt{split} divides a conjunction goal in its constituent parts.

Besides $\mathcal{L}$tac scripts, the main tool used to automate
proofs in our development is tactic \texttt{auto}. This tactic uses a
relatively simple principle: a database of tactics is repeatedly
applied to the initial goal, and then to all generated subgoals, until
all goals are solved or a depth limit is reached.\footnote{The default
  depth limit used by \texttt{auto} is 5.}  Databases to be used ---
called {\em hint databases\/} --- can be specified by command
\texttt{Hint}, which allows declaration of which theorems are part of
a certain hint database. The general form of this command is:
\begin{lstlisting}
Hint Resolve thm1 thm2 ... thmn : db.
\end{lstlisting}
where \texttt{thm}$_i$ are defined lemmas or theorems and \texttt{db}
is the database name to be used. When calling \texttt{auto} a hint
database can be specified, using keyword \texttt{with}. In Figure
\ref{simptac}, \texttt{auto} is used with database \texttt{arith} of
basic Peano arithmetic properties. If no database name is specified,
theorems are declared to be part of hint database \texttt{core}. Proof
obligations for termination are filled using lemmas
\ref{lem-termination-1}, \ref{lem-termination-2} e
\ref{lem-termination-3} that are included in hint databases. Failures
of unification, for a given list of constraints \C, is represented by
\texttt{UnifyFailure} and proof obligations related to failures are
also handled by \texttt{auto}, thanks to the inclusion of
\texttt{UnifyFailure} constructors as \texttt{auto} hints using
command 
\begin{lstlisting}
Hint Constructors UnifyFailure.
\end{lstlisting}

