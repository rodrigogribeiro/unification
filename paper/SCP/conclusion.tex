\section{Conclusion}\label{conclusion}

We have given a complete formalization of termination, soundness and
completeness of a type unification algorithm in the Coq proof
assistant. To the best of our knowledge, the proposed formalization is
the first to follow the structure of termination proofs presented in
classical textbooks on type systems~\ccite{Pierce02,Mitchell96}.
Soundness and completeness proofs of
unification are coupled with the algorithm definition and are filled
by scripted proof tactics using previously proved lemmas.

The formalized unification algorithm is used to produce a correct
constraint-based type inference algorithm for STLC in Coq. We use such
formalization to produce a Haskell implementation from it.
Since Coq extraction doesn't make use of any Haskell's library types, we use
some customization commands to produce more idiomatic Haskell code. To validate
such customizations, we have used property based tests to ensure that the
produced Haskell code behaves as expected.

The developed formalization has 962 lines of code and around 100 lines of
comments. The formalization is composed by 47 lemas and theorems,
49 type and function definitions and 5 inductive types. Most of the implementation
effort has been done on proving termination, which takes 293 lines of
our code, expressed in 21 theorems.  Compared with Kothari's
implementation, that is written in more than 1000 lines, our code is
more compact.

We intend to use this formalization to develop a complete type
inference algorithm for Haskell in the Coq proof assistant. The
developed work is available online~\ccite{unify-rep}.
