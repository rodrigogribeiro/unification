\section{A Taste of Coq Proof Assistant}\label{coq}

Coq is a proof assistant based on the calculus of inductive
constructions (CIC) \ccite{Bertot04}, a higher order typed
$\lambda$-calculus extended with inductive definitions.  Theorem
proving in Coq follows the ideas of the so-called
``BHK-cor\-res\-pon\-dence''\footnote{Abbreviation of Brouwer, Heyting,
  Kolmogorov, de Bruijn and Martin-L\"of Correspondence. This is also
  known as the Curry-Howard ``isomorphism''.}, where types represent
logical formulas, $\lambda$-terms represent proofs
\ccite{CurryHoward06} and the task of checking if a piece of text is a
proof of a given formula corresponds to checking if the term that
represents the proof has the type corresponding to the given formula.

However, writing a proof term whose type is that of a logical formula
can be a hard task, even for very simple propositions.  In order to
make the writing of complex proofs easier, Coq provides
\emph{tactics}, which are commands that can be used to construct proof
terms in a more user friendly way.

As a tiny example, consider the task of proving the following simple
formula of propositional logic:
\[
(A \to B)\to (B\to C) \to A \to C
\]
In Coq, such theorem can be expressed as:
\begin{lstlisting}
Section EXAMPLE.
   Variables A B C : Prop.
   Theorem example : (A -> B) -> (B -> C) -> A -> C.
   Proof.
       intros H H' HA. apply H'. apply H. assumption. 
   Qed.
End EXAMPLE.
\end{lstlisting}
In the previous source code piece, we have defined a Coq section named
\texttt{EXAMPLE}\footnote{In Coq, we can use sections to delimit the
  scope of local variables.} which declares variables \texttt{A},
\texttt{B} and \texttt{C} as being propositions (i.e. with type
\texttt{Prop}). Tactic \texttt{intros} introduces variables
\texttt{H}, \texttt{H'} and \texttt{HA} into the (typing) context,
respectively with types \texttt{A -> B}, \texttt{B -> C} and
\texttt{A} and leaves goal \texttt{C} to be proved. Tactic
\texttt{apply}, used with a term \texttt{t}, generates goal \texttt{P}
when there exists \texttt{t: P -> Q} in the typing context and the
current goal is \texttt{Q}. Thus, \texttt{apply H'} changes the goal
from \texttt{C} to \texttt{B} and \texttt{apply H} changes the goal to
\texttt{A}. Tactic \texttt{assumption} traverses the typing context to
find a hypothesis that matches with the goal.

We define next a proof of the previous propositional logical formula
that, in contrast to the previous proof, that was built using tactics
(\texttt{intros}, \texttt{apply} and \texttt{assumption}), is coded
directly as a function:
\begin{lstlisting}
   Definition example' : (A -> B) -> (B -> C) -> A -> C :=
       fun (H : A -> B) (H' : B -> C) (HA : A) => H' (H HA).
\end{lstlisting}
However, even for very simple theorems, coding a definition directly
as a Coq term can be a hard task. Because of this, the use of tactics
has become the standard way of proving theorems in Coq. Furthermore,
the Coq proof assistant provides not only a great number of tactics
but also a domain specific language for scripted proof automation,
called $\mathcal{L}$tac. In this work, the developed proofs follow the
style advocated by Chlipala~\cite{Chlipala13}, where most proofs are
built using $\mathcal{L}$tac scripts, to automate proof steps and make
them more robust. Details about $\mathcal{L}$tac can be found
in~\cite{Chlipala13,Bertot04}.

%We briefly illustrate these  notions by means of a small example, shown
% in Figure \ref{fig:coq-code-ex1}.
% \begin{figure}[h]
% \begin{lstlisting}
% Inductive nat : Set :=
% 	| O : nat
% 	| S : nat -> nat.

% Fixpoint plus (n m : nat) : nat :=
% 	match n with
% 	   | O => m
% 	   | S n' => S (plus n' m)
% 	end. 	  
	
% Theorem plus_0_r : forall n, plus n 0 = n.
% Proof.
% 	intros n. 
% 	induction n as [| n'].
% 	(**Case n = 0**)
% 		reflexivity.
% 	(**Case n = S n' **)
% 		simpl.
% 		rewrite -> IHn'.
% 		reflexivity.
% Qed. 	
% \end{lstlisting}
% \centering
% \caption{Sample Coq code}
% \label{fig:coq-code-ex1}
% \end{figure}
	
% The source code in Figure~\ref{fig:coq-code-ex1} shows some basic 
% features of the Coq proof assistant: types, functions and proof definitions. 
% In this example, a new inductive type is defined to represent 
% natural numbers in Peano notation. This
% type is formed by two data constructors: \texttt{O}, that represents the
% number $0$; and \texttt{S}, the successor function. For instance, in this 
% notation the number
% $2$ is represented by the term \texttt{S (S O)} of type \texttt{nat}.

% The command \texttt{Fixpoint} allows the definition of structural recursive
% functions. Function \texttt{plus} defines the sum of two unary 
% natural numbers, in a straightforward way. It is noteworthy that, in order 
% to maintain logical consistency, all functions in Coq must be total. 

% Besides the declaration of inductive types and functions, we can
% define and prove theorems in Coq. 
% Figure~\ref{fig:coq-code-ex1} 
% shows an example of a simple theorem about function \texttt{plus}, namely
% that, for an arbitrary value \texttt{n} of type \texttt{nat}, we have that 
% \texttt{plus n 0 = n}. The command \texttt{Theorem} allows 
% us to state some formula that we want to prove and it starts the 
% \emph{interactive proof mode}, in which tactics can be used to produce 
% the wanted proof term. In an interactive section of Coq (after enunciation of
% theorem \texttt{plus\_O\_r}), we must prove the following goal:
% \begin{lstlisting}
%  =============================
%    forall n : nat, plus n 0 = n 
% \end{lstlisting}
% After command \texttt{Proof.}, one can use tactics to build, step by 
% step, a term of the given type. The first tactic, \texttt{intros}, is used
% to move premisses and universally quantified variables from the goal to the
% hypothesis. Now, we need to prove:
% \begin{lstlisting}
%  n : nat
% =============================
%  plus n 0 = n 
% \end{lstlisting}
% The quantified variable \texttt{n} has been moved from the \texttt{goal}
% to the hypothesis. Now, we can proceed by induction over the structure of 
% \texttt{n}. This can be achieved by using tactic \texttt{induction}, that
% generates one goal for each constructor of type \texttt{nat}. This will
% leave us with the following two goals to be proved:
% \begin{lstlisting}
% 2 subgoals
  
% ============================
%    plus 0 0 = 0

% subgoal 2 is:
%  S n' + 0 = S n'
% \end{lstlisting}
% The goal \texttt{plus 0 0 = 0} holds trivaly by the definition of \texttt{plus}.
% Tactic \texttt{reflexivity} proves trivial equalities, after reducing both sides
% of the equality to their normal forms. The next goal to be proved is:
% \begin{lstlisting}
%  n' : nat
%  IHn' : plus n' 0 = n'
% ============================
%    plus (S n') 0 = S n'
% \end{lstlisting}
% The hypothesis \texttt{IHn'} is the automatically generated induction hypothesis for 
% this theorem. In order to finish this proof, we need to transform the goal to
% use the inductive hypothesis. To do this, we use the tactic \texttt{simpl}, which performs
% reductions based on the definition of function \texttt{plus}. This changes the goal 
% to:
% \begin{lstlisting}
%  n' : nat
%  IHn' : plus n' 0 = n'
% ============================
%    S (plus n' 0) = S n'
% \end{lstlisting}
% Since the goal now has as a subterm the exact left hand side of the hypothesis
% \texttt{IHn'}, we can use the \texttt{rewrite} tactic, which replaces some term by
% another using some equality in the hypothesis. Now, we have the following goal: 
% \begin{lstlisting}
%  n' : nat
%  IHn' : plus n' 0 = n'
% ============================
%    S n' = S n'
% \end{lstlisting}
% This can be proved immediately using the \texttt{reflexivity} tactic. 
% This tactic script builds the following term:
% \begin{lstlisting}
% fun n : nat =>
% 	nat_ind 
% 	(fun n0 : nat => n0 + 0 = n0) (eq_refl 0)
% 	  (fun (n' : nat) (IHn' : n' + 0 = n') =>
% 	   eq_ind_r (fun n0 : nat => S n0 = S n') 
% 	     (eq_refl (S n')) IHn') n
% 	     : forall n : nat, n + 0 = n
% \end{lstlisting}
% Instead of using tactics, one could instead write CIC terms directly
% to prove theorems.  This is however a complex task, even for simple
% theorems like \mbox{\texttt{plus\_O\_r}}, since the manual writing of
% proof terms requires know\-ledge of the CIC type system.  Thus,
% tactics frees us from the details of constructing type correct CIC
% terms.

% An interesting feature of Coq is the possibility of defining inductive
% types that mix computational and logic parts. This allows us to define
% functions that compute values together with a proof that this value
% has some desired property. Type \texttt{sig}, also called ``subset
% type'', is defined in Coq's standard library as:
% \begin{lstlisting}
% Inductive sig (A : Set) 
% 		(P : A -> Prop) : Set :=
% 	| exist : forall x : A, P x -> sig A P.
% \end{lstlisting}
% The \texttt{exist} constructor takes two arguments: the value \texttt{x}
% of type \texttt{A} --- that represents the computational part --- and
% an argument of type \texttt{P x} --- the ``certificate'' that the value
% \texttt{x} has the property specified by the predicate \texttt{P}. 
% As an example of a \texttt{sig} type, consider:
% \begin{lstlisting}
% forall n : nat, n <> 0 -> {p | n = S p}.
% \end{lstlisting}

% %This uses the Notation:
% %  "{ x : A | P }" := sig (fun x : A => P) 
% %That is, it is equivalent to:
% %\begin{lstlisting}
% %forall n : nat, n <> 0 -> sig (fun (S p) => n)
% %\end{lstlisting}

% This type represents a function that returns the predecessor of a natural
% number \texttt{n}, together with a proof that the returned value \texttt{p} really
% is the predecessor of \texttt{n}. Defining functions using the \texttt{sig} type
% requires writing the corresponding logical certificate. As with theorems,
% we can use tactics to define such functions. 
% \begin{lstlisting}
% Definition pred_certified : 
% 	forall n : nat, n <> 0 -> {p | n = S p}.
%    intros n H. 
%    destruct n as [| n']. 
%    (**Case n = 0**)
%    elim H. reflexivity.
%    (**Case n = S n'**)
%    exists n'. reflexivity. 
% Defined.
% \end{lstlisting}

% Using the command \texttt{Extraction pred\_certified} we can discard 
% the logical part of this function definition and get a certified
% implementation of this function in OCaml~\cite{OCaml}, Haskell~\cite{Haskell98} or Scheme~\cite{Scheme}. 
% The OCaml code of this function, obtained through extraction, is the following:

% % pred_cert below should be: pred_certified

% \begin{lstlisting}
% (** val pred_cert : nat -> nat **)
% let pred_cert = function
% 	| O -> assert false (* absurd case *)
% 	| S n0 -> n0
% \end{lstlisting}
