 \section{Definitions}\label{definitions}

\subsection{Types}\label{types}

We consider a language of simple types formed by type variables, type
constants (also called type constructors) and functional types given
by the following grammar:
\[
\begin{array}{rcl}
  \tau & ::= & \alpha\,\mid\,c\,\mid\,\tau\to\tau
\end{array}
\]
where $\alpha$ stands for a type variable and $c$ a type
constructor. All meta-variables ($\tau,\,\alpha$ and $c$) can appear
primed or subscripted and as usual we consider that $\to$ associates
to the right. 

Identifiers for variables and constructors are represented as natural
numbers, following standard practice in formalized
meta-theory~\ccite{deBruijn72,locally}. We are aware that choosing
this representation of types is not adequate to represent Haskell's
types, since it does not allow the occurrence of $n$-ary type
constructors. Using $n$-ary type constructors will only clutter
definitions due to the need of using kinds\footnote{Kinds classify
  type expressions in the same way as types classify terms. More
  details about the use of kinds and high-order operators can be found
  in~\cite{Pierce02}.}. Since the presence of kind information is
orthogonal to unification, we prefer to omit it in order to clarify
definitions and proofs.

The list of type variables of type $\tau$ is denoted by \fv$(\tau)$.

The $\size$ of a given type $\tau$, given by the number of arrows,
type variables and constructors in $\tau$, is denoted by
$\size(\tau)$. Formally:

\[ \begin{array}[t]{llp{.7\textwidth}} 
     \size(\tau_1 \to \tau_2) & = & $1 + \size(\tau_1) + \size(\tau_2)$ \\
     \size(\tau)              & = & $1$ otherwise ($\tau=\alpha$ or $\tau=c$, 
                                        for some $\alpha, c$)
   \end{array}
\]

We let $\tau_1 \eeq \tau_2$ denote the equality constraint between two
types $\tau_1$ and $\tau_2$.

Lists of equality constraints are represented by meta-variable
$\mathbb{C}$. We use the left-associative operator $::$ for
constructing lists: $a::x$ denotes the list formed by head $a$ and
tail $x$.

The definition of free type variables for constraints and their lists
are defined in a standard way and the size of constraints and
constraint lists are defined as the sum of their constituent
types. The following simple lemmas will be later used to establish
termination of the unification algorithm, defined in Section
\ref{algorithm-section}.

\begin{Lemma}\label{appsize}
For all types $\tau_1,\tau_1', \tau_2,\tau_2'$ and all lists of constraints
$\mathbb{C}$ we have that: \[\size( (\tau_1\eeq \tau_1') :: (\tau_2\eeq \tau_2') :: \mathbb{C}) 
                           < \size( (\tau_1\to\tau_2 \eeq \tau_1'\to\tau_2') :: \mathbb{C})\]
\end{Lemma}
\begin{proof}
Induction over $\mathbb{C}$ using the definition of $\size$.
\end{proof}

\begin{Lemma}\label{lengthsize}
For all types $\tau,\tau'$ and all lists of constraints $\mathbb{C}$
we have that
\[\size(\mathbb{C}) < \size( (\tau\eeq\tau') :: \mathbb{C})\]
\end{Lemma}
\begin{proof}
Induction over $\tau$ and case analysis over $\tau'$, using the
definition of $\size$.
\end{proof}

\subsection{Substitutions}\label{substitution}

Substitutions are functions mapping type variables to types. For
convenience, a substitution is considered as a finite mapping
$[\alpha_1\,\mapsto\,\tau_1,...,\alpha_n\,\mapsto\,\tau_n]$, for
$i=1,\ldots,n$, 
which is also abbreviated
as $[\overline{\alpha}\mapsto\overline{\tau}]$ ($\overline{\alpha}$
and $\overline{\tau}$ denoting sequences built from sets
$\{\alpha_1,...,\alpha_n\}$ and $\{\tau_1,...,\tau_n\}$,
respectively). Meta-variable $S$ is used to denote substitutions.

In our formalization, a mapping $[\alpha\mapsto\tau]$ is represented
as a pair of a variable and a type. Substitutions are represented as
lists of mappings, taking advantage of the fact that a variable never
appears twice in a substitution.  The domain of a substitution,
denoted by $\dom(S)$, is defined as:
\[ dom(S) = \{\alpha\,|\,S(\alpha) = \tau, \alpha\not=\tau\} \]

Following~\ccite{McBride03}, we define {\em substitution
  application\/} in a variable-by-variable way; first, let the
application of a mapping $[\alpha\mapsto\tau']$ to $\tau$ be defined
by recursion over the structure of $\tau$:

\[ \begin{array}[t]{llp{.5\textwidth}}
      \symbol{91}\alpha\mapsto \tau'\symbol{93}\,(\tau_1\rightarrow \tau_2) & = & 
        $([\alpha\mapsto \tau']\:\tau_1) \to ([\alpha \mapsto \tau']\:\tau_2)$ \\
      \symbol{91}\alpha\mapsto \tau'\symbol{93}\,\alpha & = & $\tau'$ \\
      \symbol{91}\alpha\mapsto \tau'\symbol{93}\,\tau   & = & $\tau$ 
        \text{ otherwise (\begin{tabular}[t]{l}
                                  $\tau = \alpha'$ for some $\alpha'\not=\alpha$, or \\
                                  $\tau = c$ for some $c$)
                          \end{tabular}}
   \end{array}
\]

Next, substitution application follows by recursion on the number of
mappings of the substitution, using the above defined application of a
single mapping:

\[
   S(\tau) = \left\{
             \begin{array}{ll}
               \tau & \text{ if } S = [\,]\\
              S'([\alpha\mapsto\tau']\:\tau) & \text{ if }S = [\alpha\mapsto\tau'] :: S'
             \end{array}
           \right.
\]

Application of a substitution to an equality constraint is defined in
a straightfoward way:

\[ S\:(\tau\eeq \tau') = S(\tau)\eeq S(\tau') \]

In order to maintain our development on a fully constructive ground,
we use the following lemma, to cater for proofs of equality of
substitutions. This lemma is used to prove that the result of the
unification algorithm yields the most general unifier of a given set
of types.

\begin{Lemma}\label{extsubst}
For all substitutions $S$ and $S'$, if $S(\alpha)=S'(\alpha)$ for all
variables $\alpha$, then $S(\tau) = S'(\tau)$ for all types $\tau$.
\end{Lemma}
\begin{proof}
Induction over $\tau$, using the definition of substitution
application.
\end{proof}

Substitutions and types are subject to well-formedness conditions,
described in the next section.

\subsection{Well-Formedness Conditions}

Now, we consider notions of well-formedness with regard to types,
substitutions and constraints. These notions are crucial to give
simple proofs for termination, soundness and completeness of the
unification algorithm.

Well-formed conditions are expressed in terms of a type variable
context, \V, that contains, in each step of the execution of the
unification algorithm, the {\em complement\/} of the set of type
variables that are in the domain of the unifier. This context is used
to formalize some notions that are assumed as immediate facts in
textbooks, like: ``at each recursive call of the unification
algorithm, the number of distinct type variables occurring in
constraints decreases'' or ``after applying a substitution $S$ to a
given type $\tau$, we have that $FV(S(\tau)) \cap dom(S) =
\emptyset$''.

We consider that:

\begin{itemize}

  \item A type $\tau$ is well-formed in \V, written as
    $\wf(\V,\tau)$, if all type variables that occur in $\tau$ are
    in \V.

  \item A constraint $\tau_1 \eeq
    \tau_2$ is well-formed, written as $\wf(\V,\tau_1 \eeq \tau_2)$, if
    both $\tau_1$ and $\tau_2$ are well-formed in $\V$.

  \item A list of constraints $\C$ is well-formed in
    $\V$, written as $\wf(\V,\C)$, if all of its equality constraints
    are well-formed in $\V$.

  \item A substitution $S = \{[\alpha\mapsto \tau]\}:: S'$ is well-formed in $\V$,
    written as $\wf(\V,S)$, if the following conditions apply:
     \begin{itemize}
       \item $\alpha\in\V$
       \item $\wf(\V - \{\alpha\},\tau)$
       \item $\wf(\V-\{\alpha\},S')$ 
     \end{itemize}
  \end{itemize}

The requirement that type $\tau$ is well-formed in $\V - \{\alpha\}$
is necessary in order for $[\alpha\mapsto \tau]$ to be a well-formed
substitution. This avoids cyclic equalities that would introduce
infinite type expressions.

The well-formedness conditions are defined as recursive Coq functions
that compute dependent types from a given variable context and a type,
constraint or substitution.

A first application of these well-formedness conditions is to enable a
simple definition of composition of substitutions.  Let $S_1$ and
$S_2$ be substitutions such that $\wf(\V,S_1)$ and
$\wf(\V-\dom(S_1),S_2)$. The composition $S_2 \circ S_1$ can be
defined simply as the append operation of these substitutions:
\[ S_2 \circ S_1 = S_1\: \texttt{++}\: S_2 \]
The idea of indexing substitutions by type variables that can appear
in its domain and its use to give a simple definition of composition
was proposed in~\ccite{McBride03}. 

We say that a substitution $S$ is more general than $S'$, written as
$S\leq S'$, if there exists a substitution $S_1$ such that $S' = S_1
\circ S$.  

The definition of composition of substitutions satisfies the following
theorem:

\begin{Theorem}[Substitution Composition and Application]
For all types $\tau$ and all substitutions $S_1$, $S_2$ such that
$\wf(\V,S_1)$ and $\wf(\V-\dom(S_1),S_2)$ we have that $(S_2 \circ
S_1)\,(\tau) = S_2 (S_1 (\tau))$.
\end{Theorem}
\begin{proof}
By induction over the structure of $S_2$.
\end{proof}

\subsection{Occurs Check}

\newcommand{\occurs}{\text{\it occurs\/}}

Type unification algorithms use a well-known occurs check in order to
avoid the generation of cyclic mappings in a substitution, like
$[\alpha\mapsto\alpha\to\alpha]$. In the context of finite type
expressions, cyclic mappings do not make sense. In order to define the
occurs check, we first define a dependent type, $occurs(\alpha,\tau)$,
that is inhabited\footnote{According to the BHK-interpretation, a type
  is inhabited only if it represents a logic proposition that is
  provable.} only if $\alpha\in\fv(\tau)$:

\[ \begin{array}{llp{.7\textwidth}}
      \occurs (\alpha,\tau_1\to\tau_2) & = & $\occurs(\alpha,\tau_1) \lor occurs(\alpha,\tau_2)$ \\
      \occurs (\alpha,\alpha)         & = & $\True$ \\
      \occurs (\alpha, \tau)          & = & $\False$ otherwise\\
                                      &   & $\:\:\:\:$ i.e.~if $\tau = \alpha'$ for some $\alpha'\neq\alpha$ or $\tau = c$ for some $c$
  \end{array}
\]

Coq types $\True$ and $\False$ are the unit and empty type\footnote{In
  type theory terminology, the unit type is a type that has a unique
  inhabitant and the empty type is a type that does not have
  inhabitants. Under BHK-interpretation, they correspond to a true and
  false propositions, respectively \cite{CurryHoward06}.},
respectively. Note that $occurs(\alpha,\tau)$ is provable if and only if $\alpha \in \fv(\tau)$.

Using type $occurs$, decidability of the occurs check can be
established, by using the following theorem:

\begin{Lemma}[Decidability of occurs check]
   For all variables $\alpha$ and all types $\tau$, we have that
   either $occurs(\alpha,\tau)$ or $\neg\,occurs(\alpha,\tau)$ holds.
\end{Lemma}
\begin{proof}
    Induction over the structure of $\tau$.
\end{proof}

If a variable $\alpha$ does not occur in a well-formed type, this type
is well-formed in a variable context where $\alpha$ does not occur.
This simple fact is an important step used to prove termination of
unification. The next lemmas formalize this notion.

\begin{Lemma}\label{remlemma}
For all variables $\alpha_1,\alpha_2$ and all variable contexts $\mathcal{V}$,
if $\alpha_1\in\mathcal{V}$ and $\alpha_2 \neq \alpha_1$ then
$\alpha_1 \in (\mathcal{V} - \{\alpha_2\})$.
\end{Lemma}
\begin{proof}
Induction over $\mathcal{V}$.
\end{proof}

\begin{Lemma}
Let $\tau$ be a well-formed type in a variable context $\mathcal{V}$
and let $\alpha$ be a variable such that $\neg
occurs(\alpha,\tau)$. Then $\tau$ is well-formed in $\mathcal{V}-\{\alpha\}$.
\end{Lemma}
\begin{proof}
    Induction on the structure of $\tau$, using Lemma \ref{remlemma}
    in the variable case.
\end{proof}
